\section{Conclusions}\label{Sec:Conc}

In this paper we gave a brief introduction into the lattice regression framework and its extensions. We furthermore gave an example of how covariate shift can affect credit scoring applications and proposed to use a lattice model as a mitigation strategy. In our case study we then showed that a monotonic calibrated lattice model does not perform worse on credit data then a default random forrest model. 

The proposed ameliorations in Sec. \ref{Sec:Results} can be seen as a first step to improve the study. As a next step it would be interesting to compare the model with one or more covariate shift correction algorithms presented in Sec. \ref{Sec:Relwork}.  In addition, another feature of the lattice model should be explored further: the proclaimed enhanced interpretability of the model through the lattice structure \citep{gupta2016monotonic}. Recently, interpretability is becoming a topic of concern, as so called `black box'-models are criticized for being unsuitable to be used in ethically crucial situations due to lack of transparency \citep{rudin2019stop}. 

Throughout the paper it should have shown that covariate shift touches deeply on the problem of inductive reasoning. When \citep{shimodaira2000improving} introduced the concept of covariate shift, he claimed that is it exclusively caused by model misspecification and that it would disappear for a correct model. But when do we ever know the correct specification of a model? `In reality, however, good fitting models with a modest number of parameters are hard to obtain except for very simple applications.' \citep[p.~202]{quionero2009dataset} says Shimodaira. The question `How do we justify generalization for our model from train data to test data?' will continously loom in the background.
